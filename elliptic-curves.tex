\documentclass[a4paper]{article}

\usepackage[margin=1in]{geometry}
\setlength\parindent{0pt}

\usepackage{mathtools}
\usepackage{enumitem} % enumeration label options
\usepackage{amsfonts} % \mathbb, ...
\usepackage{amsthm}

\theoremstyle{remark}
\newtheorem*{remark}{Remark}

\renewcommand{\P}{\mathbb{P}}
\newcommand{\N}{\mathbb{N}}
\newcommand{\Z}{\mathbb{Z}}
\newcommand{\Q}{\mathbb{Q}}
\newcommand{\R}{\mathbb{R}}
\newcommand{\C}{\mathbb{C}}

\title{Notes for ``Elliptic Curves'' by Vladimir Dokchitser}
\author{Calum Crossley}
\date{2023-2024}

\begin{document}

\maketitle

Preparatory information:
\begin{itemize}
    \item Books: Silverman's ``Arithmetic of Elliptic Curves''
    \item Prerequisites: basics of Galois theory, basics of number fields, basics of algebraic curves, complex analysis, $p$-adic numbers
    \item Exercise sheets: 1 per lecture, 2 out of 5 exercises for assessment
    \item Lectures: 10 of them
\end{itemize}

Tentative lecture topics:
\begin{enumerate}[label=\arabic*)]
    \item The group law
    \item Geometry over $\C$
    \item Heights
    \item The Mordell-Weil theorem
    \item Geometry over $\Q_p$
    \item Formal groups
    \item Explicit 2-descent
    \item Tate modules
    \item $L$-functions and BSD
    \item Selmer groups
\end{enumerate}

\paragraph{Pre-waffle}

This is a number theory course, so we care about solving Diophantine equations.
For example, what are the rational solutions of $x^2+y^2=1$?
\begin{equation*}
    x = \frac{2t}{t^2+1}, \quad y = \frac{t^2-1}{t^2+1}, \quad t\in\Q.
\end{equation*}
The general case is impossibly hard; it is formally undecidable. We will focus
on curves, such as one equation with two variables. Life is strongly affected by
the geometry over $\C$, where the curve is a closed orientable surface in
projective space.
\begin{itemize}
    \item Genus 0: The Riemann sphere; $\P^1$. The number theory is easy; either
        there are no $\Q$-solutions or infinitely many nicely parametrized, and
        we can decide which (Hasse principle).

    \item Genus 1 (this course): The torus. There can be no $\Q$-solutions, or
        finitely many, or infinitely many. No proven algorithm exists for the
        general case, although there are algorithms conditional on the
        Tate--Shafarevich conjecture or the BSD conjecture.

    \item Genus $\ge2$: There are finitely many $\Q$-solutions by a theorem of
        Faltings.
\end{itemize}

\begin{remark}
    By Siegel's theorem there are only finitely many $\Z$-solutions for $g\ge1$.
\end{remark}

\section{Group Law}

\end{document}
